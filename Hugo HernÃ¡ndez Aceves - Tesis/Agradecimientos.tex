\chapter{Presentación}

\textit{Se alzarán y se hundirán, como hasta ahora lo han hecho, cientos de hombres poderosos, entre reyes, tiranos y profetas, pero nunca nadie sobre ella, porque la naturaleza es la ley por encima de todas las leyes, es el poder detrás de todos los poderes, la naturaleza es lo absoluto, y cuando habla, su voz hace temblar hasta al más vehemente de los dioses. La ciencia es, por tanto, la más noble de las labores humanas en tanto arte de escuchar a la naturaleza. Una vez que el científico, es capaz de hacer frente a la terrible tentación de dejar hablar al ego, a fin de que haya suficiente silencio, será capaz de escuchar el código con el que ha sido escrito el universo, nada menos. Soy muy afortunado de tener la oportunidad de formar parte de este gran esfuerzo colectivo a través de este modesto trabajo, que más allá de que espero que pueda aportar algo de luz en la difícil lucha contra la enfermedad del HLB, simboliza para mí el más alto de los honores; mi boleto de entrada en la comunidad científica.\\
Recuerdo con mucho amor las largas sesiones en las que estudiaba álgebra o termodinámica en la silenciosa intimidad de la madrugada, ahí estábamos ella y yo, la ciencia me abrazaba desde la nada y la existencia tenía sentido, sin duda esos han sido los momentos más felices de mi vida. Y más allá del romanticismo, tampoco he de olvidar que el camino fue especialmente duro para mí, que muchas veces pensé en abandonar pero tuve la suerte de rodearme de las personas correctas, personas que fueron mi soporte para no desistir. Estudiar la carrera de física ha sido hasta ahora, la decisión más ambiciosa y osada de mi vida, elegí un camino difícil pero no me arrepiento. Afortunadamente, salimos adelante.\\
En primer lugar quiero agradecer a mis amigas. Nada hubiera sido de mí sin Maricruz Conti, una científica fuera de serie, quien literalmente me convenció de no solicitar mi baja luego del duro inicio de mi carrera. También agradezco profundamente a Adriana Tapia, mi gran amiga, quien siempre estuvo ahí. Desde luego, gracias a mi madre, que de una pieza me erigió ella sola. Gracias a Liza, mi ejemplo a seguir.\\
Gracias a la universidad pública, a la más noble todas; gracias a la BUAP, que abre las puertas de Puebla y permite a los humildes jóvenes de la región convertirse en mexicanos honorables. Gracias a mi querida Puebla; la más culta, la más aguerrida, la que desde el 5 de mayo de 1862 no ha dejado de formar a los mejores hijos de México. Gracias a la patria, que nos da el suelo, el espíritu y la voluntad.\\
Gracias a aquellos hombres cuya voluntad de verdad está por encima de sí mismos y no se reservan nunca ni un ápice de conocimiento porque saben que las ideas son un patrimonio común. Gracias por siempre a los hombres que construyen a la ciencia, a quienes entregan la vida al noble arte de descifrar «el código con el que se ha sido escrito el universo».
}



