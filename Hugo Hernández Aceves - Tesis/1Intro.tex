\chapter{Prefacio}
En las últimas décadas, la Enfermedad del Dragón Amarillo, también llamada HLB, se ha esparcido por el mundo desde Asia \cite{da2008biology}, y México ha sido fuertemente afectado por ella. Esta es una enfermedad que infecta a los árboles de cítricos y provoca que los frutos y los árboles en general se deformen y pierdan sus propiedades \cite{dala2019effect}, llevando a grandes pérdidas en las cosechas todos los años. Uno de los factores que agrava esta situación es que algunos árboles pueden ser asintomáticos durante varios meses o hasta años \cite{lee2015asymptomatic}, y esto los hace capaces de contagiar a muchos más antes de que los agricultores puedan hacer algo. Para combatir y prevenir a esta enfermedad, en el campo mexicano se implementan algunos métodos de control como el uso de pesticidas para eliminar al insecto que transmite el HLB (aquí llamado «psílido»), a través de una campaña institucional que busca reducir los contagios. Sin embargo, a pesar de todas las normas y protocolos, estos esfuerzos parecen quedar muy cortos y no ser suficientes para evitar que la enfermedad se siga extendiendo por el país, algo que no ha dejado de hacer desde su llegada. \\
Este trabajo busca emular el comportamiento de la propagación del HLB para evaluar las diversas estrategias de control y compararlas con la estrategia que actualmente se emplea. En virtud de lo anterior, en este trabajo se ha buscado evidencia, a través de la simulación computacional, del impacto del contagio asintomático, para estar en condiciones de plantear si los malos resultados en estas campañas podrían ser explicados por una consideración inadecuada de este hecho. La relación de la física con este problema, es mostrada en el siguiente capítulo.\\
En esta tesis, se ha modelado el problema usando un \textit{modelo de agentes individuales} creando un programa basado en el sistema que forman los árboles  y los psílidos involucrados en el HLB en México, posteriormente, se han puesto a prueba distintas variables típicas de una huerta de esta naturaleza; como los métodos de control utilizados y la cantidad inicial de psílidos, y a través de este método se ha estudiado la dinámica de los contagios.\\
En este trabajo, se ha observado que: Uno. Aunque la aplicación de pesticidas es de lejos el método que más retrasa la infección, en promedio hasta 260 días, no la contiene del todo. Dos. Los métodos que no consideran el efecto asintomático, como en el método de la remoción de árboles, tienen un impacto insignificante en el control. Tres. En todos los casos, la propagación de psílidos, psílidos infecciosos y árboles infectados, tuvieron forma de manchas radiales y concéntricas. Esto implica que la forma de obtener muestras, tendrá que tener este efecto en cuenta, como el «método en T», del que se habla en el capítulo tercero. Cuatro. Se ha visto que sin importar las condiciones iniciales ni los métodos de control empleados, el comportamiento del crecimiento de la población de árboles sintomáticos determina el comportamiento del surgimiento de árboles asintomáticos.



%Intro vieja
%En las últimas décadas la Enfermedad del Dragón Amarillo, también llamada HLB, se ha esparcido por el mundo desde Asia, y México ha sido fuertemente afectado por ella. Esta es una enfermedad que aqueja a los árboles de cítricos y provoca que los frutos sean incomestibles, llevando a la eventual muerte de toda la planta. El problema se agrava dado que algunos árboles pueden ser asintomáticos, de modo que son capaces de contagiar a muchos más antes de que los agricultores puedan hacer algo. Para combatir y prevenir a esta enfermedad, en el campo mexicano se implementan cada año algunos métodos de control como el uso de pesticidas para eliminar al insecto que transmite el HLB, además existen normas y protocolos institucionales para monitorear los contagios. A pesar de todo, los esfuerzos parecen quedar muy cortos y no ser suficientes para evitar que se pierda parte de las cosechas.
%Este trabajo parte de la hipótesis de que hay grandes áreas de oportunidad en la forma en la que institucionalmente se ha combatido el HLB, y busca proponer, a través de la simulación computacional, mejores formas de aplicar las estrategias de control del insecto transmisor de la Enfermedad del Dragón Amarillo, así como estrategias para detectar árboles contagiados asintomáticos y prevenir que contagien a más árboles.
%Lo que se ha hecho en esta tesis es crear una simulación por computadora del sistema que forman los árboles cítricos y los psílidos que transmiten el HLB, posteriormente se han puesto a prueba dentro del programa, algunos métodos de control y se ha estudiado cómo la enfermedad se propaga y qué factores parecen inhibirla y qué otros la potencian.
%Se ha encontrado que la normativa se puede mejorar en cuanto al tratamiento de los árboles asintomáticos. Este no es un tema menor dado que México es uno de los principales productores de cítricos a nivel mundial y la repercusión económica de las mermas es importante, de modo que las mejoras en las normas implicarían también la mejora de la producción cítrica del país.

