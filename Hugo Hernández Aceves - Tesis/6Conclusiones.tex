\chapter{Conclusiones}

Este capítulo sintetiza los resultados descritos en el anterior. De este trabajo se han desprendido cuatro grandes afirmaciones. En primer lugar, en lo que respecta a los distintos métodos de control, y cómo estos afectan a la dinámica de crecimiento de los árboles asintomáticos, especialmente reduciendo su propagación, se encontró que no existe mejor herramienta de entre las evaluadas, para combatir el crecimiento de la infección, que el uso recurrente de pesticidas. Cuanto mayor mortalidad y frecuencia tengan estos pesticidas, más lento avanza la enfermedad. Este método no precisa el conocer ningún tipo de información sobre la huerta, basta con aplicarlo uniformemente en ella para obtener, de lejos, mucho mejores resultados que en aquellos métodos que sí requieren de estudiar el campo. Finalmente, en consistencia con lo observado en la realidad, este método está lejos de ser infalible, pues sólo sirve para retrasar la propagación, en el mejor de los casos hasta 260 días, pero no sirve para detenerla. Por otro lado, un método que sí se vale de información recabada de la huerta, como lo es el eliminar árboles, tiene un impacto menor, debido a que sólo se aplica a los árboles sintomáticos, cuando la gran mayoría son asintomáticos.

El segundo resultado de interés es el que respecta a la dinámica de los psílidos infecciosos, cuya propagación sigue a la del resto de psílidos salvo por un tiempo de desfase, esto es, que si se estudia en el tiempo la trayectoria del crecimiento de las poblaciones de psílidos, se verá que la trayectoria de los psílidos infecciosos imitará a la del total de psílidos. Además, hay una clara relación entre las regiones con árboles infectados y las regiones con psílidos infecciosos, llegando a ser idénticas las siluetas de estas dos poblaciones. La explicación que se propone es que en las zonas de la huerta con árboles infectados habrá la mayoría de psílidos infecciosos, y que los que lleguen a emigrar a otro árbol, al ser una minoría, no contribuyen en gran medida a formar la mancha de psílidos infecciosos, y luego, cuando algún psílido infeccioso llega a infectar a un árbol, este comenzará a infectar a todos sus habitantes. Es por esta razón que las zonas con psílidos infecciosos y árboles infectados, son casi equivalentes. También se propone que la mancha de psílidos infecciosos está desfasada de la del resto de psílidos porque esta obedece al comportamiento de la mancha de árboles infectados, y a su vez estos árboles infectados no crecen directamente con las dispersión de psílidos sanos, sino que tardan cierto tiempo en infectarse en función de la cantidad de psílidos que alojen y del tiempo que lleven haciéndolo. Esto último explica por qué, en condiciones ideales, la enfermedad se propaga de forma radial, y justifica la siguiente conclusión.
 
Cuando se toman muestras de árboles en alguna región infectada, el método para hacerlo es el «método en t», que se ha descrito detalladamente en el capítulo tercero. Este método comienza tomando muestras a partir de algún árbol que se considere el punto de partida de la infección, y desde él se avanza radialmente hacia los demás árboles en las cuatro direcciones. Este método está justificado porque en las condiciones de esta simulación, la propagación de los psílidos y de las infecciones crece como manchas concéntricas. La última conclusión es la relativa a la relación entre árboles asintomáticos y sintomáticos. Se ha mostrado anteriormente que sin importar los métodos de control y las condiciones de la huerta, el comportamiento del crecimiento de árboles asintomáticos está condicionado siempre por el de los árboles sintomáticos. Esta es una dirección en la que esta investigación puede continuar, buscando una forma rigurosa de calcular la cantidad de árboles asintomáticos a partir de los datos de árboles sintomáticos, de tal forma que sea posible estimar la cantidad de árboles asintomáticos a voluntad.

